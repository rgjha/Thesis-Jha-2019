\thispagestyle{empty}
\begin{center}
{\large {\bf VITA}}
\end{center}
\vspace{1cm}
{\bf AUTHOR}: Raghav Govind Jha \footnote{raghavgjha.net} \\
{\bf NATIONALITY}: Indian \\
{\bf DATE OF BIRTH}: January 23, 1989 \\
{\bf DEGREES AWARDED}:
\begin{itemize}
\item M.S Physics, Syracuse University, Syracuse, NY  (2015)
\item M.Sc. Physics, St. Xavier's College, University of Calcutta, India (2013) 
\item MS in Nanotechnology \& Material Science, UPMC, Paris, France (2011) 
\item B.Sc in Physics with Honors, St. Stephen's College, University of Delhi, India (2010) 
\end{itemize}
{\bf PROFESSIONAL EMPLOYMENT}:
\begin{itemize}
\item[] Graduate Teaching/Research Assistant, Department of Physics, Syracuse University (2013-2019).
\end{itemize}
{\bf PUBLICATIONS \footnote{$^{\bigstar}$ = Journal article, $^{\blacklozenge}$ = Conference proceedings}}:
 \begin{enumerate}
  \item Tensor renormalization group study of the non-Abelian Higgs model in two dimensions [\textbf{\textcolor{blue}{\href{https://arxiv.org/abs/1901.11443}{1901.11443}}}]  $^{\bigstar}$
 \item  Lattice quantum gravity with scalar fields [\textbf{\textcolor{blue}{\href{https://arxiv.org/abs/1810.09946}{1810.09946}}}] $^{\blacklozenge}$
  \item  The properties of $D1$-branes from lattice super Yang--Mills theory using gauge/gravity duality   [\textbf{\textcolor{blue}{\href{https://arxiv.org/abs/1809.00797}{1809.00797}}}] $^{\blacklozenge}$
  \item On the removal of the trace mode in lattice $\mathcal{N }= 4$ super Yang-Mills theory  [\textbf{\textcolor{blue}{\href{https://arxiv.org/abs/1808.04735}{1808.04735}}}] \href{https://journals.aps.org/prd/accepted/8d072Q69Ide18d24f2c61259a36a4536c02c700b5}{(accepted in Phys Rev. D)} $^{\bigstar}$
 \item Nonperturbative study of dynamical SUSY breaking in $\mathcal{N}$ = (2, 2) Yang-Mills theory  [\href{https://journals.aps.org/prd/abstract/10.1103/PhysRevD.97.054504}{Phys.\ Rev.\ D {\bf 97}, 054504 (2018)}] [\textbf{\textcolor{blue}{\href{https://arxiv.org/abs/1801.00012}{1801.00012}}}]   $^{\bigstar}$
 \item Truncation of lattice $\mathcal{N}$ = 4 super Yang-Mills [\href{https://doi.org/10.1051/epjconf/201817511008}{EPJ Web of Conferences 175, 11008 (2018)}] $^{\blacklozenge}$
\item Testing the holographic principle using lattice simulations  [\href{https://doi.org/10.1051/epjconf/201817508004}{EPJ Web of Conferences 175, 08004 (2018)}] [\textbf{\textcolor{blue}{\href{https://arxiv.org/abs/1710.06398}{1710.06398}}}] $^{\blacklozenge}$
\item Testing holography using the lattice with super-Yang-Mills theory on a 2-torus [\href{https://journals.aps.org/prd/abstract/10.1103/PhysRevD.97.086020}{Phys.\ Rev.\ D {\bf 97}, 086020 (2018)}] [\textbf{\textcolor{blue}{\href{https://arxiv.org/abs/1709.07025}{1709.07025}}}] $^{\bigstar}$

\end{enumerate}


{\bf AWARDS}:
 \begin{itemize}
 \item \emph{Henry Levinstein Fellowship for Outstanding Senior Graduate Student} \hfill 2017 
\item Henry Levinstein Fellowship - Department of Physics, Syracuse University \hfill 2014
 \item CSIR/UGC-NET - Junior Research Fellowship (JRF) by Government of India \hfill 2013
 \item Erasmus Mundus Scholarship for pursuing M.S at University of Paris VI \hfill 2010
 \item NIUS Fellowship by TIFR, Mumbai \hfill 2008
 \item KVPY Scholarship by Department of Science \& Technology, Government of India \hfill 2008
 \item Awarded the Merit certificate by University of Delhi (Top 25 students in the university out of total number of students $\approx$ 1000) \hfill 2008
 \end{itemize}


{\bf TEACHING}:
 \begin{itemize}
  \item Recitation Instructor for PHY 216 (General Physics II for Honors and Majors) and grader for  \hfill Spring 2019 \\ PHY 662 (Quantum Mechanics II)  \hfill 
  \item Recitation Instructor for PHY 215 (General Physics I for Honors and Majors) and grader for  \hfill Spring 2018 \\ PHY 312 (Relativity \& Cosmology)  \hfill 
 \item Grader for PHY 424 (Electromagnetism) and PHY360 (Waves and Oscillations)  \hfill Fall 2016
 \item Recitation Instructor for PHY 212 General Physics II \hfill Spring 2016
 \item Grader for PHY 641 (Statistical Mechanics) and \\ PHY731 (Electromagnetic theory)  \hfill Spring 2015
 \item Recitation Instructor for PHY 211 General Physics I \hfill Spring 2014, Summer 2014, Fall 2014
 \end{itemize}



{\bf PRESENTATIONS}:
\textcolor{red}{\textbf{Invited Talks}} :   
 \begin{itemize}
  \item Holographic dualities and tensor renormalization group study of gauge theories (March 11, 2019) [1 hour] at Perimeter Institute, Waterloo, Canada 
  \href{http://www.perimeterinstitute.ca/videos/interdisciplinary-seminar-holographic-dualities-and-tensor-renormalization-group-study-gauge}
{\textcolor{blue}{[Video of the talk at PIRSA]}} 
  \end{itemize}
  \begin{itemize}
  \item Supersymmetry breaking and gauge/gravity duality on the lattice (April 6, 2018) [25+5 minutes] \href{[http://www-hep.colorado.edu/~eneil/lbsm18/talks/Jha.pdf]} at 
  \href{http://www-hep.colorado.edu/~eneil/lbsm18/}{\textcolor{blue}{LBSM 2018, UC Boulder, Colorado, USA}} 
  \end{itemize} 
 
  \begin{itemize}
  \item Recent results from lattice supersymmetry in 2 $\le$ d $<$ 4 dimensions (January 31, 2018) [25+5 minutes] [\href{https://www.youtube.com/watch?v=Zey6DAEiw0c}{Video}] at 
  \href{https://www.icts.res.in/program/NUMSTRINGS2018}{\textcolor{blue}{NUMSTRINGS - Bangalore, India}} 
  \\
  \emph{(Simon Catterall was invited for this talk)} 
  \end{itemize} 
  
 
 \begin{itemize}
 \item Testing holography through lattice simulations (April 4, 2017) [40+5 minutes]  at 
 \href{https://sites.google.com/site/kyotoquantumgravity2017/home/program}{\textcolor{blue}{Quantum Gravity, String Theory and Holography Workshop - Kyoto, Japan}} \\
 \emph{(Simon Catterall was invited for this talk)} 
 \end{itemize}
 
 

\begin{itemize}
\item Supersymmetry on the lattice (April 17, 2016) [30+5 minutes]  at 
\href{https://absuploads.aps.org/presentation.cfm?pid=11807}{\textcolor{blue}{April Meeting 2016 - Salt Lake City, Utah, USA}}  \\
 \emph{(David Schaich was invited for this talk)}
\end{itemize}
  
 
\textcolor{magenta}{\textbf{Contributed Talks}} :  

 \begin{itemize}
  \item Testing holographic principle through lattice studies (June 22, 2017) [15+5 minutes]  at \href{https://makondo.ugr.es/event/0/session/96/contribution/50}{\textcolor{blue}{Lattice 2017 - 35th International Symposium on Lattice Field Theory}}   
  \item Lattice quantum gravity with scalar fields (July 23, 2018) [15+5 minutes]  at \href{https://indico.fnal.gov/event/15949/session/15/contribution/80}{\textcolor{blue}{Lattice 2018 - 36th International Symposium on Lattice Field Theory}} 
 \end{itemize}
 
 
 \textcolor{orange}{\textbf{Posters}} :  

  \begin{itemize}
 \item \href{https://indico.fnal.gov/event/15949/session/4/contribution/66}{The properties of D1-branes from lattice super Yang--Mills theory using gauge/gravity duality} at Lattice 2018

 \end{itemize}