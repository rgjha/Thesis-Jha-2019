\epigraph{\textit{I think of my lifetime in Physics as divided into three periods. In the first period ...I was in the grip of the idea that 
Everything is Particles...I call my second period Everything is Fields...Now I am in the grip of a new vision that...Everything 
is Information\\} \vspace{3mm} - John A. Wheeler} 

The main theme of the thesis was to explore super-Yang-Mills (SYM) theories in various
dimensions on the lattice while preserving a fraction of supersymmetry at finite lattice spacing. 
The lattice simulations of maximally supersymmetric theories play an important role in understanding 
various aspects of holography in detail by directly dealing with strongly coupled field theories. 
Though there have been numerous calculations on the gravity side 
predicting properties of strongly coupled field theories, the other direction has been relatively 
unexplored. The real power and understanding of the holographic principle is expected to come 
when we study field theories at finite couplings to understand the nature of stringy corrections in the 
supergravity side when it is no longer classical supergravity. To initiate such a program in $d >1$, 
we studied the phase diagram of the SYM theory in two dimensions and located the deconfinement 
transition which is related to the transition between two different black hole solutions. 
The horizon topology for four-dimensional black holes is fixed, but when we consider black holes
 in higher dimensions (such as in Type II supergravity theories), the horizon can have topology 
 change which is holographically dual to the deconfinement transition in the gauge theory at finite 
 temperatures. 

The primary goal is to study the four-dimensional conformal $\mathcal{N}=4$ SYM 
since most of the holographic predictions are for that theory, but this has been a major obstacle
in the past few years even after a series of improvements in the lattice formulation of $\mathcal{N}=4$ SYM. 
In addition, the numerical calculations of the four-dimensional theory are very expensive in the 
large $N$ limit even with a parallel code over the lattice volume. It is conceivable that
parallelizing over $N$ will be possible in the future and will considerably improve our access to 
supergravity limits using lattice calculations. 
The lower-dimensional SYM theories ($d > 4$) which take part in holography do not 
have the sign problem when fermions obey anti-periodic boundary conditions around the thermal
cycle. This is encouraging for Monte Carlo simulations though it must also be mentioned that the sign problem 
becomes a major obstacle in the $\lambda > 5$ studies of SYM in four dimensions with large $N$. 
It seems that going to strong couplings in four dimensions would require some new and path-breaking ideas. 
However, it is certainly possible, as we have shown, to extract interesting Physics from 
lower-dimensional SYM theories. The twisted lattice action for SYM theories in $d <4$ can still 
be studied for interesting holographic applications and 
we will continue working on it in coming years. Apart from the thermodynamics of the dual 
$D2$-branes which is work in progress, we also hope to reproduce the prediction for the static 
potential in this three dimensional maximally supersymmetric Yang-Mills theory. In order to 
explore other numerical approaches to gauge theories (than Monte Carlo), 
we have recently studied a two-dimensional non-Abelian gauge theory \cite{Bazavov:2019qih}
which is not part of the thesis.  

On the other side, there has been a lot of progress in numerical studies of
the critical systems in $d \le 2$ and their holographic implications using the 
tensor network constructions. The tensor networks, such as MERA \cite{2008PhRvL.101k0501V} 
are conjectured to capture important aspects of holography and offers non-perturbative 
insights into the geometry of the bulk through the entanglement of the quantum state 
\cite{Swingle:2009bg, 2015PhRvL.115t0401E, 2011JSP...145..891E, 
2018RvMP...90c5007N, VanRaamsdonk:2009ar, Headrick:2018ctr}.

One of the recent ideas in holography has been the proposal of holographic entanglement entropy (HEE) 
by Ryu-Takayanagi (RT) in 2006 for time-independent geometries. As an example, 
they considered a time-slice (constant time) of $AdS_{3}$ and showed that the entropy matched the 
one calculated in two-dimensional CFT. 
It was extended to time-dependent geometries by Hubeny, Rangamani, and Takayanagi (HRT) 
\cite{2007JHEP...07..062H}.
The RT formula says that 
\begin{equation}
S_{A} = \frac{\text{min.}(\text{Area}(\gamma_{A}))}{4G_{N}} \Bigg \vert_{\partial \gamma = \partial A} 
\end{equation} 
The equation tells us to consider only those minimal curves $\gamma$ which satisfies
the homology \footnote{One can think about this in terms of 2-sphere $\mathbb{S}_{2}$. 
Jordan curve theorem says that any cycle on the sphere can be shrunk to a point. In this 
sense, all cycles/curves on this manifold satisfy homology condition} condition (\emph{i.e.} 
those which can be continuously deformed) such that boundary of $\gamma$ is the same as 
that of $A$. And in case, there is more than one such minimal curve, we choose the one with 
the smallest area. 
For example, consider a black hole (where regions are not simply connected). 
If we consider that A is the entire-space time, then the boundary of that is an empty set. 
How does one choose a minimal surface in this situation? $\gamma_{A}$ cannot be empty set since the 
space-time is not connected. It turns out that the minimal surface is the event horizon. 
Note that minimal surface is just a curve for time-independent $AdS_{3}$.
So, for this case, the HEE formula is just the familiar Bekenstein-Hawking entropy formula. 
If we consider four-dimensional SYM theories on $\mathbb{S}^{3}$ 
at finite temperatures and large $N$, they are expected to undergo Hawking-Page phase 
transition which is dual to deconfinement transition as discussed above. When the field 
theory is at zero temperature, the entanglement entropy of the subsystems A $\&$ B is 
the same. At finite temperatures, the area of minimal curves change and we get a difference 
in entropy. It is also expected that entanglement entropy can serve as an order parameter. 

It will be very interesting to compute the EE on the lattice for supersymmetric theories
in lower dimensions in the future. This is a very exciting and interesting time for different numerical 
approaches to holography and understanding the features of quantum gravity working at the intersection 
of high energy theory, quantum information theory, and condensed matter theory.

