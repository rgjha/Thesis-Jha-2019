\section*{\label{app:1RG} \noindent Appendix A.1 Operators and their relevance}
If an operator \cO behaves as $E^{\Delta_{i}}$, then $\Delta_{i}$ is its dimension and $g_{i}$ has units of $E^{D - \Delta_{i}}$, where $D$ is the spacetime dimension. For ex : 
\beq
\frac{1}{2} \int d^D x \partial_{\mu} \phi \partial ^{\mu} \phi 
\eeq
$\phi$ has dimension $E^{D/2 - 1}$. Hence an operator constructed out of A $\phi$'s and B derivatives has 
\beq
\Delta_{i} = A \Bigg (\frac{D}{2} - 1 \Bigg) + B 
\eeq
We can define dimensionless couplings as : 
\beq
\lambda_{i} = \Lambda^{\Delta - D} g_{i} 
\eeq
Note that when $ \Delta_{i} = D$, then $g_{i}$ is dimensionless and we don't need any contruction. 
\bea
\int d^{D} x \cO \sim E^{D - \Delta_{i}} \\ 
\eea 
so that $i^th$ term is of order (in $g_{i}$) 
\beq
\Bigg (\frac{E}{\Lambda} \Bigg)^{\Delta_{i} - D} \lambda_{i} 
\eeq
We see that if $\Delta_{i} > D$, then the term becomes less and less important at low energies. 
If $\Delta_{i} < D$, it becomes more and more important at low energies like $\cN = (8,8)$ SYM in two dimensions. 
The results are shown in Table (\ref{tab:RG1}). 
\begin {table}
 \renewcommand\arraystretch{1.9}   % Increase the height of each row
  \addtolength{\tabcolsep}{2 pt}    % Increase separation between columns
\begin{center}
\begin{tabular}{ |  p{2cm} |  p{2cm} || p{2cm} |c || p{4cm}}
    \hline
    $\Delta_{i}$ & Size in IR & Nature & Property  \\ \hline  \hline
     < D  & Increases  &  Relevant & Super-renormalizable \\ \hline 
     > D  & Decreases &  Irrelevant & Non-renormalizable \\ \hline 
     = D  & Constant & Marginal & Strictly renormalizable \\ \hline
\end{tabular}
\vspace{3mm}
\caption {\label{tab:RG1}Dimension of operators, size in the IR, nature and property of the theory.} 
\end{center}
\end {table}




\section*{\label{app:2spinors} Appendix A.2 Spinors in various dimensions}

We have often come across notations like $\mathcal{N} = (2,2)$ and $\mathcal{N} = (8,8)$ in the main text. This is different from the notation we are used to in four dimensions. In 4d, the Weyl representation is complex, so that the representation of $\overline{Q}$ is fixed to be the conjugate of the Q representation. In 2d, the Weyl representation is real ($\textit{Majorana-Weyl}$) and the $\overline{Q}$ representation is independent of the Q representation. This can be seen for various dimensions from Table (\ref{tab:spinor}). 

\begin {table}[htbp] 
\begin{center}
\begin{tabular}{ |  p{3cm} | l | l |  l | p{3cm} |}
    \hline
    \textbf{Dimensions(d)} & \textbf{Majorana} & \textbf{Weyl} & \textbf{Weyl-Majorana}  & \textbf{Min.Rep} \\ \hline \hline 
     2 & Yes & Self & Yes & 1 \\ \hline 
     3 & Yes & - & - & 2 \\ \hline
     4 & Yes & Complex & - & 4 \\ \hline
     5 & - & - & - & 8 \\ \hline
      6 & - & Self & - & 8 \\ \hline
     7 & - & - & - & 16 \\ \hline
     8 & Yes & Complex & - & 16 \\ \hline
     9 & Yes & - & - & 16 \\ \hline
     10 (\textit{mod 8}, 2) & Yes & Self & Yes & 16 \\ \hline
     11 (\textit{mod 8}, 3) & Yes & - & - & 32 \\ \hline
     12 (\textit{mod 8}, 4) & Yes & Complex & - & 64 \\ \hline
\end{tabular}
\vspace{3mm}
\caption {\label{tab:spinor} Dimensions in which various conditions are allowed for $SO(d-1,1)$ spinors} 
\end{center}
\end {table}


\section*{\label{app:3sugra} Appendix A.3 Type II SUGRA details}
When the consider the black hole solutions at finite temperature, the temperature, $T_{H}$ is given by : 
\begin{equation}
\label{eq:temp_SUGRA}
T \propto \frac{(7-p) U_{0}^{(5-p)/2}}{4 \pi \sqrt{d_{p} \lambda}}
\end{equation}
for a uniform black $p$-brane solution. The $ \alpha^{\prime}$ corrections can be written in terms of $U_{0}$ as, 
\begin{equation}
  \label{eq:alpha_prime}
\alpha^{\prime} R \propto \sqrt{\frac{U_{0}^{(3-p)}}{\lambda}}
\end{equation}
We see from \ref{eq:temp_SUGRA} and \ref{eq:alpha_prime} that $ \alpha^{\prime} R \propto t^{\frac{3-p}{5-p}}$\footnote{The dimensionless temperature $t$, is constructed 
from $T$ by combining with appropriate powers of $\lambda$} and hence, $ (\alpha^{\prime})^3$ correction depends on $t^{\frac{9-3p}{5-p}}$. This power counting has been discussed in \cite{Berkowitz:2016jlq}.


%%%%%%%%%%%%%%%%%%%%%%%%%%%%%%

\section*{\label{app:4energy} Appendix A.4 Free energy results for SYM theories}

Some useful integrals in the study of thermodynamics of weak coupling limit of SYM theories are, 
\begin{equation}
    \int_{0}^{\infty} \frac{x^{d-1}}{e^{x} -1} dx = \Gamma(d)\zeta(d) 
\end{equation}

\begin{equation}
    \frac{7}{8}\int_{0}^{\infty} \frac{x^3}{e^{x} -1} dx = 
    \int_{0}^{\infty} \frac{x^3}{e^{x}+1} dx = \frac{7\pi^4}{120} 
\end{equation}

\begin{equation}
    \frac{3}{4}\int_{0}^{\infty} \frac{x^2}{e^{x} -1} dx = 
    \int_{0}^{\infty} \frac{x^2}{e^{x}+1} dx = \frac{3 \zeta(3)}{2}
\end{equation}

\begin{equation}
    \frac{1}{2}\int_{0}^{\infty} \frac{x}{e^{x} -1} dx = 
    \int_{0}^{\infty} \frac{x}{e^{x}+1} dx = \frac{\pi^2}{12}
\end{equation}
In (3+1)-dimensions, the fermionic degrees of freedom contribute 7/8 of bosonic ones. In sixteen supercharge case, 
we get,  
\begin{equation}
N^2 \left[ 8 + (7/8)8 \right] = 15N^2
\end{equation}
In general, total d.o.f in $d$ dimensions will be, 
\begin{equation}
8 \Big( 2 - \frac{1}{2^{d-1}}\Big) N^2
\end{equation}
Black body radiation gives 
\begin{equation} 
E = \frac{VT^4 \pi^2}{30} ~~~\text{per d.o.f}
\end{equation} 
For $\mathcal{N}=4$ SYM in four dimensions, we have 
\begin{equation} 
E = 15N^2 \frac{VT^4 \pi^2}{30} 
\end{equation} 
and the entropy is,  
\begin{equation} 
S = \frac{2\pi^2}{3} VT^3N^2 
\end{equation} 
The free energy is given by, 
\begin{equation}
F = \frac{-\pi^2}{6} VT^4N^2 
\end{equation} 
%This has also be  by Klebanov and Tsetylin at zero $\lambda$. 
%Also, see Zweibach Page 557
A similar analysis in 3d gives, 
\begin{equation}
E =  \frac{14\zeta(3)}{\pi}VN^2T^3 
\end{equation} 
The entropy is given by,  
\begin{equation}
S = \frac{21\zeta(3)}{\pi}VN^2T^2
\end{equation} 
and free energy F is ($g_{\text{YM}} \to 0$) 
\begin{equation}
F = \frac{-7\zeta(3)}{\pi}VN^2T^3
\end{equation} 
%Matches arxiv 9905030 
The free energy density for ${\cal N}=4$  SYM in the weak coupling regime is :
\begin{equation}
    f = \frac{F}{V} = - \left(4 + 2n_{s} + \frac{7}{2} n_{f}\right) N^{2} \zeta(4) T^{4} \frac{1}{V} 
\end{equation}
Here, V is the unit volume of three sphere $S^{3}$ = $2\pi^2$. 
Two interesting limits to this expression :
\begin{itemize}
\item Only gauge: $n_{s} = 0$, $n_{f} = 0$ gives $f = - \frac{\pi^2}{45} N^{2}T^{4}$, which is the photon gas result! 
\item Full theory: $n_{s} = 6$, $n_{f} = 4$ gives $f = - \frac{\pi^2}{6} N^{2}T^{4}$
\end{itemize}

%%%%%%%%%%%%%%%%%%%%%%%%%%%%%%


\section*{\label{app:5dforms} Appendix A.5 Differential forms}

The lattice construction of super Yang-Mills theories on the lattice makes use of special kind of fermions which are 
integer forms rather than spinors. Here, we will briefly review the important results of differential forms. 
See ~\cite{Zumino:1983rz} ~\cite{MTW} for further details. 
A scalar function is called a \emph{0-form} and is defined as : 
\beq
df \equiv \frac{\partial f}{\partial x^{\mu}} dx^{\mu} 
\eeq 
Suppose we have a vector function $\Phi_{\mu}$, we construct a \emph{1-form} $\Phi$ as $ \Phi = \phi_{\mu} dx^{\mu}$. 
and define, 
\beq
d \Phi \equiv \frac{\partial \Phi_{\mu}}{\partial x^{\nu}} dx^{\nu} \wedge dx^{\mu} 
\eeq 
The $\wedge$ denotes the wedge product defined as : 
\beq
dx^{\nu} \wedge dx^{\mu}  = - dx^{\mu}  \wedge dx^{\nu}
\eeq
We can interpret $d\phi$ as curl of $\Phi$. 
Generally, from an anti-symmetric tensor with p-indices, 
we can construct a \emph{p-form} as, 
\beq
\Phi = \Phi_{\mu_{1}, \mu_2, \cdots \mu_{p}} \Bigg( \frac{1}{p!} dx^{\mu_1} \wedge dx^{\mu_2} \wedge \cdots dx^{\mu_p} \Bigg)  
\eeq
It is obvious that we cannot have p-forms with $p > D$, where D is the number of dimensions. 
\beq
d\Phi = \partial_{\nu} \Phi_{\mu_{1}, \mu_2, \cdots \mu_{p}} \Bigg( \frac{1}{p!} dx^{\nu} \wedge dx^{\mu_1} \wedge dx^{\mu_2} \wedge \cdots dx^{\mu_p} \Bigg)  
\eeq
Let's take a \emph{p-form} $\alpha$ and \emph{q-form} $\beta$, 
then we have 
\beq
\alpha \beta = (-1)^{pq} \beta \alpha 
\eeq 
Then, 
\beq
d(\alpha \beta) = (d\alpha)\beta + (-1)^{p}\alpha d(\beta)
\eeq 
In YM theory, the gauge potential is \emph{1-form} 
\beq
A = A_{\mu} dx^{\mu} 
\eeq 
and, field tensor F is defined schematically as \footnote{$F = \frac{1}{2} (\partial_{\mu} A_{\nu} - \partial_{\nu} A_{\mu} + [A_{\mu}, A_{\nu}]) dx^{\mu} dx^{\nu}$} 
\beq
F = dA + A^{2} 
\eeq 
with, 
\bea
dF & = d(dA + A^{2})  \\ 
     & = d(AA) \\
     & = d(A)A - AdA \\
\eea 
Also, 
\beq
[A, F] = [A, dA] = AdA - dAA 
\eeq
Adding above two, we get (Bianchi identity from forms) 
\beq
DF \equiv dF + [A, F ] = 0 
\eeq
where, $D (\cdot) = d (\cdot) + [A, (\cdot)]$. 


\section*{\label{app:6pfaffian} Appendix A.6 Note on Pfaffian}

Let M be a complex $d \times d$ matrix which is anti-symmetric (also known as skew-symmetric, $M^{T} = -M$), then we have $\text{det}M = 0$, if d = odd. Therefore, we will assume that $d = 2n$. 
For an even-dimensional complex $ 2n \times 2n$ matrix, pfaffian is defined as, 
\begin{equation}
\text{pf(M)} = \frac{1}{2^n n!} ~ \epsilon_{i_{1}}\epsilon_{j_{1}} \cdots \epsilon_{i_{n}}\epsilon_{j_{n}}, 
\end{equation}
where, $\epsilon$ is the alternating tensor of rank 2n and sum over 
repeated indices is assumed. There also exists alternative definition, 
without normalization factors written as, 
\begin{equation}
\text{pf(M)} = \sum_{P} (-1)^{P}  M_{i_{1}}M_{j_{1}} \cdots M_{i_{n}}M_{j_{n}}, 
\end{equation}
Here, P is the set of permutations of $\{i_{1}, i_{2}, \cdots i_{2n}\}$ with respect to  $\{1, 2, \cdots, 2n\}$ such 
that $ i_{1} < j_{1} \cdots i_{n} < j_{n} \cdots i_{2n} < j_{2n} $ and $ i_{1} < i_{2} \cdots i_{2n}$. P take values 
$\pm 1$ for even and odd permutations. Note that if M is odd-dimensional matrix, pf(M) = 0. 
Some corresponding theorems :
\begin{itemize}
\item For an arbitrary, $ 2n \times 2n$ matrix, Q, and complex anti-symmetric 
$ 2n \times 2n$ matrix, P, we have : $\text{pf}(QPQ^{T}) = \text{pf}(P) \text{det}(Q)$. 
\item If M is a complex anti-symmetric matrix, then $\text{det} M = [\text{pf}(M)]^{2}$.
\end{itemize}
The relation between pfaffian and determinant is : $\text{pf}(M) = \pm \sqrt{\text{det}M}$. The sign is of utmost importance for our purposes and is 
determined by the correct branch of the square root. 
We can also define these in terms of path integral over Grassmann variables. 
Given an antisymmetric $ 2n \times 2n$ matrix M and $2n$ real 
Grassmann variables $\eta_{i}$ where i = $\{1, 2, \cdots, 2n\}$, the pfaffian of M is given by :
\begin{equation}
\text{pf}(M) =  \int  d\eta_{1} d\eta_{2} \cdots d\eta_{2n} ~ \text{exp} \Bigg( \frac{-1}{2}\eta_{i} M_{ij} \eta_{j}\Bigg)
\end{equation}
Given an $ n \times n$ complex matrix A and $n$ pairs of complex Grassmann variables $\psi_{i}$ and $\overline{\psi_i}$ 
where i = $\{1, 2, \cdots, n\}$ 
\begin{equation}
\label{eq:det_formula}
\text{det}(A) =  \int  d\overline{\psi}_{1} d\psi_{1} \cdots d\overline{\psi}_{n} d\psi_{n} d\overline{\psi}_{n} ~ \text{exp}  \Big( \overline{\psi}_{i} A_{ij} \psi_{j}\Big)
\end{equation}
From the definition of pfaffian, we have
\begin{align}
\label{pfaffian0}
[\text{pf}(M)]^{2}&=  
\int  d\eta_{1} d\eta_{2} \cdots d\eta_{2n} ~ \text{exp} \Big( \frac{-1}{2}\eta_{i} M_{ij} 
\eta_{j}\Big) \int  d\chi_{1} d\chi_{2} \cdots d\chi_{2n} ~ \text{exp} \Big( \frac{-1}{2}\chi_{i} M_{ij} \chi_{j}\Big)\\
                         &= \int  d\eta_{1} d\eta_{2} \cdots d\eta_{2n}  \int  d\chi_{1} d\chi_{2} 
                         \cdots d\chi_{2n} ~ \text{exp} \Big( \frac{-1}{2}\eta_{i} M_{ij} \eta_{j} -  \frac{-1}{2}\chi_{i} M_{ij} \chi_{j}\Big)\\
                         &=(-1)^{n(2n+1)} \int d\chi_{1} d\eta_{1}\int d\chi_{2} d\eta_{2} \cdots 
                         \int d\chi_{2n} d\eta_{2n} ~ \text{exp} \Big( \frac{-1}{2}\eta_{i} M_{ij} \eta_{j} -  \frac{-1}{2}\chi_{i} M_{ij} \chi_{j}\Big) \\
\end{align}
We can now define, 
\begin{equation}
\psi_{i} \equiv \frac{1}{\sqrt{2}} (\chi_{i} + i \eta_{i})
\end{equation}
and $\overline{\psi}$. We have 
\[ \int d\chi_{1} d\eta_{1}\int d\chi_{2} d\eta_{2} \cdots \int d\chi_{2n} d\eta_{2n}  \to (-1)^{n} d\overline{\psi}_{1}\psi_{1} \cdots d\overline{\psi}_{2n}\psi_{2n} \]
using $ (-i)^{2n} = (-1)^n$, where -i is the Jacobian of the change of variables. Then we get, 
\begin{equation}
[\text{pf}(M)]^{2} =  \int  d\overline{\psi}_{1} d\psi_{1} \cdots d\overline{\psi}_{n} d\psi_{n} d\overline{\psi}_{n} ~ \text{exp}  \Big( \overline{\psi}_{i} A_{ij} \psi_{j}\Big)
\end{equation}
which is equal to \ref{eq:det_formula}. 


















