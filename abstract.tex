Lattice studies of strongly coupled gauge theories started with the pioneering work of 
Wilson. The success of lattice QCD since then has improved our understanding of the 
strong dynamics, crucial for proper understanding of many interesting phenomena in Physics. 
However, it is now known that the Standard model is only an approximation to some richer 
underlying theory. It is believed that supersymmetry has a special role to play in the framework of that theory. 
Even if nature is non-supersymmetric at all energy scales and we see no experimental evidence 
in the coming decades, the beautiful structure of these theories
and the great failure could still be very important lessons in our quest to understand the universe. 
In four dimensions, a special supersymmetric theory has drastically altered our understanding 
of the holographic principle. In view of these observations, the study of supersymmetric
gauge theories on lattice at strong couplings is crucial. Even though lattice supersymmetry has a long history 
going back four decades, it has been very difficult to simulate the four-dimensional theory at strong
couplings till date. This is because supersymmetry on the lattice is far from trivial and is broken 
at the classical level because of the supersymmetric algebra. However, substantial progress has 
been made in studying these theories on the lattice. Several wonderful ideas like topological twisting, 
differential forms, point group symmetries of the lattice, and integer form fermions all come together 
and has enabled us to study these supersymmetric theories by preserving a subset of 
supersymmetries exactly on the lattice. This thesis deals with the numerical studies of super Yang-Mills 
(SYM) theories in various dimensions, their large $N$ limit, and their role in a better understanding 
of gauge/gravity duality. 